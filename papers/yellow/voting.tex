\section{Voting}

Many parameters can be changed on-the-fly via miners voting, such as instructions costs, computational cost limit per block,
block size limit, storage fee factor, block version, and so on. Voting for the block version~(so for a soft-fork)
lasts for 32 epochs~(see epoch length below), and requires more than 90 percent of the miners to vote for the change.
For less critical changes~(such as block size limit), simple majority is enough. We will further refer to the changes
of the first kind as to foundational changes, and we refer to the changes of the second kind as to everyday changes.
Per block, a miner can vote for two everyday changes and also one foundational change, with the votes to be included
into the header of the block.

To vote "Yes"~("I'm agree on the change proposed"), and also to propose change in the first block of an epoch, a miner
is publishing identifier of the change directly in the block header. To vote "No" (or avoid voting at all, which is
the same), a miner is simply writing zero value instead of a corresponding byte (another option is to provide a vote
identifier which is not being considered within the epoch).

System constants:
\begin{itemize}
\item{} Voting epoch length = 1024 blocks.
\item{} Voting epochs per foundational change = 32
\item{} Voting epochs before approved foundational change activation = 128
\end{itemize}

\subsection{Parameters table}
\label{sec:params-table}

The following table describes vote identifiers, default values (during launch), possible steps, and also minimum and maximum values.
If the step is not defined in the table, its value is defined as $\max(\lfloor current\_value / 100 \rfloor, 1)$.
If minimum value for a parameter is not defined, it equals to zero. If maximum value is not defined, it equals to
1,073,741,823.

To propose or vote for increasing a parameter, a miner is inluding a parameter identifier ($id$) into the block header.
If the miner is for decreasing parameter, he is including ($-id$) into the block header.

% DON'T EDIT THIS TABLE.
% it is auto-generated by org.ergoplatform.tools.DefaultParametersPrinter

\begin{tabular}{*{6}{l}}
    Id & Description & Default & Step & Min & Max \\
    \hline
    1 & Storage fee factor (per byte per storage period) & 1250000 & 25000 & 0 & 2500000 \\
    2 & Minimum monetary value of a box & 360 & 10 & 0 & 10000 \\
    3 & Maximum block size & 524288 & - & 16384 & - \\
    4 & Maximum cumulative computational cost of a block & 1000000 & - & 16384 & - \\
    5 & Token access cost & 100 & - & - & - \\
    6 & Cost per one transaction input & 2000 & - & - & - \\
    7 & Cost per one data input & 100 & - & - & - \\
    8 & Cost per one transaction output & 100 & - & - & - \\
    120 & Soft-fork (increasing version of a block) & - & - & - & - \\
\end{tabular}

Parameter values are to be written into extension section on first block of a voting epoch,
that is, in the extension of a block when its $height\,mod\,1024 = 0$.
Parameters for initial moment of time~$(height = 1)$ are simply hardcoded.

\subsection{Proposing a change and voting for it}

To propose a change, in the first block of a voting epoch (of $1,024$ blocks, so in a block of
$height\,mod\,1024 = 0$), a miner is posting an identifier of a vote for a change. There are three slots (three bytes)
in a block header for changes to propose, with two slots reserved for everyday changes and the third one for
proposing a softfork. Slot not occuppied by a proposal is to be set to zero. Votes could come in any order.
Examples of the bytes: $(0, 1, 120)$, $(120, -3, 0)$. In the first case, a miner is proposing to increase storage fee factor ($id:1$), and
also proposes a soft-fork ($id:120$), In the second case, a miner is proposing to decrease block size ($id:-3$), and also
 is proposing a soft-fork ($id:120$).

To vote for a proposal~(which is, again, proposed in the first block of an epoch) within the epoch, a miner is including vote identifier
into the block header. Identifiers not proposed in the first block of the epoch are ignored.

If majority of votes within an epoch are supporting an everyday change (so at least 513 blocks are containing an
identifier), a new value of the parameter should be written into the extension section of the first block of the next
epoch.

\subsection{Voting for a soft-fork}

A soft-fork happens when a protocol version supported by the network is being increased.
The protocol version is being written into every block header, with initial value during launch to be set to 1.

Semantics behind versioning is not defined ahead of time by the protocol and so up to clients and their developers.
During a soft-fork, protocol developers are able do disable some existing validation rules from~\ref{sec:validation},
while also introduce new ones.

The protocol version upgrade is to be done in the following steps:
\begin{enumerate}
    \item{} A protocol developer implements and releases a software with changed rules.
    This software may deactivate some of existing rules, while also introduce new logic.
    \item{} A miner is proposing to increase protocol version and put deactivated rules
    into extension.
    \item{} Other miners are voting within 32 epochs for the proposal
    \item{} If the soft-fork proposal is being rejected~(so if it gathers no more than $90\%$ of votes during voting period, i.e.
            no more than $\lfloor 32 * 1024 * 90 / 100 \rfloor = 29491$ votes), new voting may be proposed next epoch
            after the voting done.
    \item{} If the soft-fork proposal is being approved, activation period of 32 epochs is starting. The first block
    after the activation period is called activation height. It is prohibited to vote for a new soft-fork during the
    activation and also on the activation height. Soft-fork data is still to be
    written to corresponding extension sections~(see below) during the activation period, and on activation height.
    Block version written into extensions is to be increased from the first block of the activation period, while protocol
    version in headers is still the same. Protocol version in headers is increased from the activation height.
\end{enumerate}


To start voting for a soft-fork, a miner needs to publish the identifier $120$ in the first block of the epoch, consider,
for example, that its height is $h_s$. Next epoch, a miner should post height when start fork was proposed ($122: h_s$)
in the extension section of the block, and number of votes collected in the previous epochs $v_s$ should be written
there as well as ($121: v_s$).

Examples:

\begin{itemize}
    \item{} Assume that a vote for soft-fork is proposed in block number $1024$ (by writing a record into the extension
            with a key equals to $120$ and any value). Assume that within this epoch,
            so before block number $2048$, $500$ votes were collected. Thus a miner which is generating block
            $\#2048$ must write height when the soft-fork voting has been started in the extension section of the block
            as ($122: 1024$ key-value pair), as well as number of votes supporting the fork gathered within the epoch
            done (in form of $121: 500$). Assume that the new epoch started with the block $\#2048$ brings 600 votes for
            fork. Then a miner generating the block $\#3072$ must write down the following pairs there: $(121, 1100)$ and
            $(122, 1024)$.
    \item{} The voting is to be done on height $1024 + 1024*32 = 33792$, so the last vote could be written into the
            block $\#33791$. Block $\#33792$ still should contain correct values for votes collected and voting starting
            height.
    \item{} Lets assume that voting for the soft-fork was not successful, i.e. no more than $29491$ votes for soft-fork
            have been gathered between blocks $\#1024$~(inclusive) and $\#33792$~(exclusive). Then the next epoch block,
            so the block $\#34816$ should clear the old voting parameter values. A new voting for soft-fork may be
            started in this new epoch, so in the block $\#34816$, if the block contains a vote for the soft-fork
            (i.e. a record with key = 120 in the extension section), then the block must contain new starting height
            and votes collected values (i.e. following pairs: $(121: 0)$ and $(122: 34816)$). If there is no new voting
            started, all the parameters regarding soft-fork voting must be cleared off of the block $\#34816$.
    \item{} Now assume that the voting for the soft-fork was successful, i.e. more than $29491$ votes for soft-fork
            have been gathered between blocks $\#1024$~(inclusive) and $\#33792$~(exclusive). In this case, activation
            epoch is starting immediately after voting has been done, so on height $33792$. Activation epoch lasts for
            32 epochs, so the first block after activation period has height of $33792$. Please note that any vote
            for a soft-fork during the activation period is prohibited. Also, on a first block of a voting epoch
            soft-fork voting parameters should be written~(height when voting had been started and also number of votes
            collected). On the first block after activation, protocol version written into a block must be increased.
            The first block after activation should carry soft-fork voting parameters. The parameters must be cleared
            next epoch, so in a block $\#33792$. A new voting may be started in the block $\#33792$, similarly to the
            case of non-successful voting.
\end{itemize}
