\documentclass[]{article}

\usepackage{graphicx}
\usepackage{amssymb}
\usepackage{color}
\usepackage{hyperref}
\usepackage{float}

\bibliographystyle{abbrv}

\newcommand{\knote}[1]{{\textcolor{green}{Alex notes}}{#1}}
\newcommand{\dnote}[1]{{\textcolor{red}{Dima notes}}{#1}}

\begin{document}
    \title{Ergo: The Decentralized Platform For Contractional Money}
    %    \author{Dmitry Meshkov}

    %    \date{January 25, 2019\\v1.0}

    \maketitle

    \begin{abstract}
        \dnote{todo}
    \end{abstract}

    %    #######################################################################################
    %    Базовые мысли:
    %
    %    Основные фичи - децентрализация и контракционные деньги
    %    Изначальные пользователи - фанаты децентрализации кто уже более-менее в теме, и устали от проблем BTC/ETH (пример - Justin, mx)
    %
    %    #######################################################################################


    \section{Introduction}

    %    history of Ergo
    %    problems of BTC/ETH
    %    Why did we started Ergo

    \section{Social contract}

    Ergo protocol is very flexible and may be changed in the future by the community.
    In this section we define the main principles, that should be followed during the Ergo protocol updates.
    In case of intentional violation of any of these principles, the resulting protocol should not
    be called Ergo.

    \begin{itemize}
        \item{\em Decentralization first.} Ergo should be as decentralized as possible.
        This means that any parties (social leaders, software developers, hardware manufacturers, miners and so on)
        which absence or malicious behaviour may affect the security of the network should be avoided.
        \item{\em Created for regular people.} Ergo is the platform for regular people, their interests should
        not be infringed upon in favor of big players. In particular, that means that they should be able to
        participate in the protocol by running a full node and mine blocks (with small probability).
        \item{\em Platform for contractional money.} Ergo is the base layer to applications, that will be
        build on top of it. It allows to implement any kind of applications, but it's main goal is
        to provide efficient, secure and easy way to implement financial contracts. \dnote{probably kushti may write this better}
        \item{\em Long terms focus.} All aspects of Ergo development should be focus on long-term perspective.
        At any point of time, Ergo should be able to survive for centuries without expected hardforks,
        software or hardware improvements or some other unpredictable changes. As far as Ergo is oriented
        to be a platform, applications built on top of Ergo should also be able to survive in a long term.
        \item{\em Permissionless and open.} Ergo protocol does not restrict or limit any categories of usage.
        It should allow anyone to join the network and participate the protocol without any preliminary permissions.
        No bailouts, blacklists or other forms of discrimination should be possible on core level of Ergo protocol.
        On the other hand application developers are free to implement any logic they want, taking responsibility
        for the ethics and legality of their application.
    \end{itemize}


    \section{Survivability}

    %   well-tested solutions
    %   voting
    %   soft-forkability
    %   storage rent
    %   light clients (сослаться на раздел ниже)

    \section{Consensus}

    %   Почему выбрали PoW
    %   Известные проблемы PoW
    %   Детали Автоликуса

    \section{Clients}

    %   Пробелмы пользователей без легких клиентов
    %   nipopow/flight clients
    %   Дайджест узлы

    \section{Economy}

    %   пишет Саша
    %   ценность нашего токена, и филосовские размышления о данном типе экономики

    \section{Contractional Money}

    %   пишет Саша

    %   практические удобные и эффективные контракты для денег
    %   Описать про UTXO модель, почему мы ее выбрали
    %   Скрипт - доступ к стейту, доступ к блокчейну, цепочки (тут можно сослаться на Эфир, ибо там написано что в UTXO это невозможно), сигма протоколы
    %   token system - токены на базовом уровне, сильно упрощают протоколы вида "token threshold", позволяют разграничить потоки
    %   ? Анонимность и пример контракта миксинга
    %   Примеры 1-2 контракта, который иллюстрируют несколько основных преимуществ эрго

    \section{Conclusions}

    \bibliography{references}

    \section{Appendix A: Coins emission}

    %    подробная информация об эмиссии, скрипту эмиссии, и всему-всему что нужно для фаундейшена

\end{document}
