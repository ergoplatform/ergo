\knote{this section is to be rewritten and split into protocol and implementation parts}

\section{Blockchain synchronization}

Ergo modifiers can be in one of the following states:

\begin{itemize}
    \item{\em Unknown} - synchronization process for corresponding modifier is not started yet.
    \item{\em Requested} - modifier was requested from another peer.
    \item{\em Received} - modifier was received from another peer, but is not applied to history yet.
    \item{\em Held} - modifier was held by NodeViewHolder - PersistentModifiers are held by History,
    Ephemeral modifiers are held by Mempool.
    \item{\em Invalid} - modifier is permanently invalid.
\end{itemize}

The goal of the synchronization process is to move modifiers from \term{Unknown} to \term{Held} state.
In the success path modifier changes his statuses \term{Unknown}->\term{Requested}->\term{Received}->\term{Held},
however if something went wrong (e.g. modifier was not delivered) it goes back to \term{Unknown} state
(if node is going to receive it in future) or to \term{Invalid}state (if node is not going to receive
this modifier anymore).

\subsection{From \term{Unknown} to \term{Requested}}

Modifier can go from \term{Unknown} state to \term{Requested} one by different ways, that depend on
current node status (bootstrapping/stable) and modifier type.

\textbf{Inv protocol}

\term{Inv} (inventory) message contains a pair: \term{(ModifierTypeId, Seq[ModifierId])}.
When one node sends \term{Inv} message to another one, it is assumed that this node
contains modifiers with the specified ids and type and is ready to send on request.

Node broadcasts \term{Inv} message in 2 cases:
\begin{itemize}
    \item - When it successfully applies a modifier to \term{History} and modifier is new enough.
    This is useful to propogate new modifiers as fast as possible when nodes are already synced with the network
    \item - When it receives \term{ErgoSyncInfo} message (see \textbf{Headers synchronization} for more details)
\end{itemize}

When node received \term{Inv} message it
\begin{itemize}
    \item - filter out modifiers, that are not in state \term{Unknown}
    \item - request remaining modifiers from the peer that sent \term{Inv} message.
    Modifier goes into \term{Requested} state.
\end{itemize}

\textbf{Headers synchronization}

First, node should synchronize it's headers chain with the network.
In order to achieve this every \term{syncInterval} seconds node calculates \term{ErgoSyncInfo} message,
containing ids of last \term{ErgoSyncInfo.MaxBlockIds} headers and send it to peers,
defined by function \term{peersToSyncWith()}.
If there are outdated peers (peers, which status
was last updated earlier than \term{syncStatusRefresh} seconds ago) \term{peersToSyncWith()} return outdated peers,
otherwise it returns one random peer which blockchain is better and all peers with status Unknown
\dnote{peersToSyncWith() logic is not intuitive, it's better to write description, why this choice?}.

To achieve more efficient synchronization, node also sends \term{ErgoSyncInfo} message every time when headers chain is
not synced yet, but number of requested headers is small enough (less than $desiredInvObjects / 2$).

On receiving \term{ErgoSyncInfo} message, node calculates \term{OtherNodeSyncingStatus},
which contains node status (\term{Younger}, \term{Older}, \term{Equal}, \term{Nonsense} or \term{Unknown}) and extension -
\term{Inv} for next \term{maxInvObjects} headers missed by \term{ErgoSyncInfo} sender.
After that node sends this \term{Inv} message to \term{ErgoSyncInfo} sender.

\textbf{Block section synchronization}

After headers application, a node should synchronize block sections
(BlockTransactions, Extension and ADProofs), which amount and composition
may vary on node settings (node with UTXO state does not need to download ADProofs,
node with non-negative \term{blocksToKeep}
should download only block sections for last \term{blocksToKeep} blocks, etc.).

In order to achieve this, every \term{syncInterval} seconds node calculate \term{nextModifiersToDownload()} -
block sections for headers starting at height of best full block, that are in \term{Unknown} state.
These modifiers are requested from random peers (since we does not know a peer who have it),
\footnote{we can keep a separate modifierId->peers map for modifiers, that are not received yet and try to download from this peers first}.
and they switch to state \term{Requested}.

To achieve more efficient synchronization, node also requests \term{nextModifiersToDownload()} every time when
headers chain is already synced and number of requested block sections is small enough (less than $desiredInvObjects / 2$).

When headers chain is already synced and node applies block header, it return \term{ProgressInfo} with \term{ToDownload} section,
that contains modifiers our node should download and apply to update full block chain.
When NVS receives this ToDownload request, it requests these modifiers from random peers
and these modifiers goes to state \term{Requested}.

\subsection{From \term{Requested} to \term{Received}}

When our node requests a modifier from other peer, it puts this modifier and
corresponding peer to special map \term{requested} in \term{DeliveryTracker} and sends \term{CheckDelivery} to self
with \term{deliveryTimeout} delay.

When a node receives modifier in \term{requested} map
(and peer delivered this modifier is the same as written in \term{requested}) -
NVS parse it and perform initial validation.
If modifier is invalid (and we know, that this modifier will always be invalid) NVS penalize peer and move modifier to state \term{Invalid}.
If the peer have provided incorrect modifier bytes (we can not check,
that these bytes corresponds to current id) penalize peer and move modifier to state \term{Unknown}.
If everything is fine, NVS sends modifier to NodeViewHolder(NVH) and set modifier to state \term{Received}.

When \term{CheckDelivery} message comes, node check for modifier - if it is already in \term{Received} state,
do nothing.
If modifier is not delivered yet, node continue to wait it up to \term{maxDeliveryChecks} times, and after that
penalize peer (if not requested from random peer) and stop expecting after that (modifier goes to \term{Unknown} state).

\subsection{From \term{Received} to \term{Held}}

When NVH receives new modifiers it puts these modifiers to modifiersCache and
then applies as much modifiers from cache as possible.
NVH publish \term{SyntacticallySuccessfulModifier} message for every applied modifier and when NVS receives this message it moves modifier to state \term{Held}.
If after all applications cache size exceeds size limit, NVH remove outdated modifiers from cache and publish
\term{ModifiersProcessingResult} message with all just applied and removed modifiers.
When NVS receives \term{ModifiersProcessingResult} message it moves all modifiers removed from cache without application to state \term{Unknown}.